\documentclass[10pt]{beamer}

\usepackage{amsmath}
\usepackage{amssymb}
\usepackage{bm}
\usepackage{hyperref}
\usepackage{csquotes}

\usetheme[progressbar=frametitle]{metropolis}
\usepackage{appendixnumberbeamer}

\usepackage{booktabs}
\usepackage[scale=2]{ccicons}

\usepackage{pgfplots}
\usepgfplotslibrary{dateplot}

\usepackage{xspace}
\newcommand{\themename}{\textbf{\textsc{metropolis}}\xspace}


\DeclareSymbolFont{matha}{OML}{txmi}{m}{it}% txfonts
\DeclareMathSymbol{\varv}{\mathord}{matha}{118}
\usetheme{metropolis}

%\usefonttheme{serif} % default family is serif

\setbeamertemplate{section in toc}[sections numbered]
\setbeamertemplate{subsection in toc}[subsections numbered]

\title{Article presentation :}
\subtitle{Chancel L., Rehm Y., The Carbon Footprint of Capital: Evidence from France, Germany and the US based on Distributional Environmental Accounts}
\author{COMPERAT Étienne  \\ GUGELMO CAVALHEIRO DIAS Paulo \\ ORTIZ Marie Ange}
\institute{Sciences Po}
\date{\today}

\newcommand\ReduceFont{\fontsize{10}{7.2}\selectfont}

\begin{document}

\begin{frame}
    \titlepage
\end{frame}

\begin{frame}
    \ReduceFont
    \frametitle{Outline}
    \tableofcontents[hideallsubsections]
\end{frame}

\section{Introduction}
\begin{frame}
    \tableofcontents[currentsection, hideothersubsections, sections=\value{section}]
\end{frame}

\subsection{Carbon Footprint}
\begin{frame}{\subsecname}
    \begin{columns}
        \begin{column}{0.5\textwidth}
            \includegraphics[width=1\linewidth, height=0.4\textheight]{../Figures/Camp.png}
            \includegraphics[width=1\linewidth, height=0.4\textheight]{../Figures/Camp2.png}
        \end{column}
        \begin{column}{0.5\textwidth}
            \textbf{Carbon Footprint:} The measure of the exclusive total amount of emissions of carbon dioxide that is directly and indirectly caused by an activity or is accumulated over the life-cycle stages of a product. \\
            \textbf{Individual Carbon Footprint:} The carbon footprint associated with an individual’s activities, lifestyle or choices. \\
            \textbf{Challenge:} What to include in the carbon footprint? \\
            The Consumption-based Approach
        \end{column}
    \end{columns}
\end{frame}

\subsection{Carbon Footprint of Capital : Chancel and Rehm (2024)}
\begin{frame}{\subsecname}
    \begin{center}
    "The Carbon Footprint of Capital: 
    \textit{Evidence from France, Germany and the US based on Distributional Environmental Accounts}"
    \end{center}

    \textbf{Motivations:}
    \textit{Individuals are not only responsible for their consumption, but also for the assets they own.}
    \begin{enumerate}
        \item Linking carbon emissions to asset ownership to construct a new framework for individual carbon footprint (3 Approaches: Consumption, Ownership and Mixed).
        \item Applying this framework to France, Germany and the US.
        \item Deriving new stylized facts about emissions inequality in the context of environmental and tax policy.
    \end{enumerate}
\end{frame}

\subsection{Key findings}
\begin{frame}{\subsecname}
    \begin{enumerate}
        \item Carbon inequalities are notable in every approach.
        \item In the ownership approach, the majority of emissions for the wealthiest 10\% originates from the assets they own.
        \item Emissions from capital ownership appear to be even more concentrated than capital itself.
    \end{enumerate}
    \includegraphics[width=1\linewidth]{../Figures/F1.png}
\end{frame}

\section{Related Literature}
\begin{frame}
    \tableofcontents[currentsection, hideothersubsections, sections=\value{section}]
\end{frame}

\subsection{Measuring the Carbon Footprint}
\begin{frame}{\subsecname}

    \textit{What makes a good Carbon Footprint estimate?}
    
    The 2 fundamentals or carbon accounting:
        \begin{enumerate}
            \item \textbf{Comprehensiveness:} measuring both direct and indirect emissions associated with the economic activity.
            \item \textbf{Exclusivity:} no double-counting.
        \end{enumerate}
    
    Together, these conditions guarantee the \textbf{macro-consistency} of the measures. \\
    So far, the two common ways to measure the carbon footprint have been to focus on \textbf{countries and firms} or \textbf{individuals} (as final consumers).
\end{frame}

\subsection{Consumption-based Approaches}
\begin{frame}{\subsecname}
    \textbf{Individuals' consumption guide the resource allocation in the economy.}
        \begin{itemize}
            \item Underlying assumption: \textit{"Individuals express their preferences through consumption, which sens a signal to producers about what to manufacture and in what quantity."}
            \item The \textbf{"consumer-pays"} principle
        \end{itemize}
        \textbf{Advantage:} Particularly relevant at the country level (accounts for outsourced emissions). \\
        \textbf{Drawback:} Puts the entire responsibility for all emissions on final consumers (despite market failures: lack of information, agency or alternatives).
\end{frame}

\subsection{Production-centered Approaches and Methods of Shared Attribution}
\begin{frame}{\subsecname}
    Contrasting consumption footprints with the production footprints of firms. \\
    \hfill \break
    \textbf{Production-centered approaches}: Focusing on the firm level \\
    Critique: Firms operate through human intervention and individuals are behind their behaviors. $\rightarrow$ \textbf{Ownership-based approach} \\
    \hfill \break
    \textbf{Methods of shared attribution}: Split emissions between consumers and firm owners \\
    Critique: Hard to implement at the individual level. $\rightarrow$ \textbf{Mixed-based approach} \\
    \hfill \break
    \textbf{Income-based carbon accounting}: An alternative at the individual level?
\end{frame}

\subsection{From the Carbon Footprint of individual investment portfolios to the DINA}
\begin{frame}{\subsecname}
    \textit{There already were some attempts at measuring the carbon emissions of individual portfolios (GHG, PCAF). But there exists no consensus regarding these methods and their estimates were not always consistent with aggregate estimates.}
    \vspace{15pt}
    
    Their answer:
    \textbf{the Distributional National and Environmental Accounts (DINA)}
    \begin{itemize}
        \item Goal of the DINA framework: \textit{Reconciling macroeconomic studies (e.g., production, income, wealth) with microeconomic distributional analysis by integrating the study of inequality into the system of national accounts. }
    \end{itemize}
\end{frame}

\section{Data Sources and Methodology}
\begin{frame}
    \textbf{Key Datasets Used:}
    \begin{itemize}
        \item \textbf{Wealth Data:} HFCS (France/Germany) and DINA (US).
        \item \textbf{Macroeconomic Data:} National accounts from Eurostat and OECD.
        \item \textbf{Emissions Data:} Air emission accounts (Eurostat, OECD).
        \item \textbf{Input-Output Tables:} Eurostat FIGARO dataset (2010–2020).
    \end{itemize}

    \textbf{Example Insights:}
    \begin{itemize}
        \item Emissions in US agriculture: \textbf{534.9 tCO$_2$/million dollars owned}.
        \item Manufacturing emissions in France: \textbf{95.1 MtCO$_2$} (ownership approach).
    \end{itemize}
\end{frame}

\section{Carbon Footprint of the Capital}
\begin{frame}{\secname}
    \tableofcontents[currentsection, hideothersubsections, sections=\value{section}]
\end{frame}

\subsection{Capital emissions by industry and institutional sector}

\begin{frame}{\subsecname}
    \begin{columns}
        \begin{column}{0.5\textwidth}
            Industries :
                \begin{itemize}
                    \item Agriculture and mining
                    \item Energy, water and waste
                    \item Manufacturing
                    \item Transport
                    \item Real estate and construction
                    \item Health and education
                    \item Public administration
                    \item Services
                \end{itemize}
        \end{column}
        \begin{column}{0.5\textwidth}
            \ReduceFont
            Results : 
                \begin{itemize}
                    \item Manufacturing as the largest emitting sector in FR and DE
                    \item Agriculture and mining as the largest emitting sector in the US
                    \item Agriculture and mining as the most carbon-intensive sector
                    \item Similar carbon intensity for the manufacturing sector
                    \item Difference in definition for the Real Estate and Construction sector
                \end{itemize}    
        \end{column}
    \end{columns}
    \hfill \break
    Following : Table 1, Emission intensities by industry groups
\end{frame}

\begin{frame}{\subsecname}
    \begin{center}
        \includegraphics[height=0.9\textheight]{../Figures/T1.png}    
    \end{center}
\end{frame}

\subsection{Capital emissions by asset class}

\begin{frame}{\subsecname}
    \begin{columns}
        \begin{column}{0.5\textwidth}
            Assets :
                \begin{itemize} 
                    \item Housing assets
                    \item Business assets
                    \item Equities 
                    \item Pension assets
                    \item Fixed income assets
                \end{itemize}
        \end{column}
        \begin{column}{0.5\textwidth}
            \ReduceFont
            Results : 
                \begin{itemize}
                    \item Equity is the most polluting asset class.
                    \item Pension assets are the second most polluting asset class.
                    \item Business assets are the third most polluting asset class.
                    \item Housing has an important market valuation, but emits little.
                    \item Important intensity of pension assets for Germany.
                \end{itemize}    
        \end{column}
    \end{columns}
    \hfill \break
    In clear, there exist important differences between types of assets.
\end{frame}

\begin{frame}{\subsecname}
    \begin{center}
        \includegraphics[width=\textwidth]{../Figures/T2.png}    
    \end{center}
\end{frame}

\subsection{The role of foreign capital in national emissions}
\begin{frame}{\subsecname}
    \begin{itemize}
        \item In France and in the US, equity held abroad represents about 20-25\% of owned equities.
        \item In Germany, equity held abroad represents about 40\% of owned equities.
        \item Foreign equity held by French and German citizens are more carbon intensive than those owned by the US citizens.
    \end{itemize}
\end{frame}

\section{The Distribution of Carbon Footprints}
\begin{frame}{\secname}
    \tableofcontents[currentsection, hideothersubsections, sections=\value{section}]
\end{frame}

\subsection{Emissions rise with income and wealth}
\begin{frame}{\subsecname}
    \begin{columns}
        \begin{column}{0.5\textwidth}
            Generally :
            \begin{itemize}
                \item Emissions are positively correlated with wealth.
                \item Consumption approach : carbon inequalities are less concentrated than income.
                \item Mixed-based approach : carbon inequalities are as concentrated as income.
                \item Ownership approach : carbon inequalities are more concentrated than wealth.
            \end{itemize}
        \end{column}
        \begin{column}{0.5\textwidth}
            \ReduceFont
            International comparison :
            \begin{itemize}
                \item The US are more carbon inequal than Germany, which is more carbon inequal than France.
                \item The majority of the US emit as much as the top of the distribution of France and Germany in the two first approaches.
                \item The top French group emits less despite owning more of the national equity than their German counterpart.
            \end{itemize}
        \end{column}
    \end{columns}
\end{frame}

\begin{frame}{\subsecname}
    \includegraphics[width=\textwidth]{../Figures/F51.png}
\end{frame}

\begin{frame}{\subsecname}
    \includegraphics[width=\textwidth, height=0.9\textheight]{../Figures/F52.png}
\end{frame}

\subsection{Emissions intensity rises with wealth}
\begin{frame}{\subsecname}
    Average emission intensity tends to increase alongside with wealth at the very top of the distribution.
    This explains the greater concentration of carbon emissions compared to wealth.
    \begin{center}
        \includegraphics[width=0.9\textwidth]{../Figures/F6.png}
    \end{center}
\end{frame}

\subsection{The weight of capital emissions among top groups}

\begin{frame}{\subsecname}
    \begin{itemize}
        \item Importance of the emissions of top groups.
        \item Emissions of the top 1\% (p.36) :
        \begin{table}[!ht]
            \centering
            \begin{tabular}{|l|l|l|l|l|}
            \hline
                Countries & Consumption & Ownership & Multiplication in tCO2e \\ \hline
                France & 2.5\% & 21.5\% & 6 \\ \hline
                Germany & 2\% & 22.3\% & 11 \\ \hline
                US & 6.2\% & 26.9\% & 16 \\ \hline
            \end{tabular}
        \end{table}
        \item Key role of Capital ownership in the determinant of the top of the distribution.
        \item Structure of the emissions alongside the wealth distribution.
    \end{itemize}
\end{frame}

\begin{frame}{\subsecname}
    \begin{center}
        \includegraphics[width=0.9\textwidth]{../Figures/F71.png}
    \end{center}
\end{frame}

\begin{frame}{\subsecname}
    \begin{center}
        \includegraphics[width=0.9\textwidth]{../Figures/F72.png}
    \end{center}
\end{frame}

\section{Discussion}
\begin{frame}{\secname}
    \tableofcontents[currentsection, hideothersubsections, sections=\value{section}]
\end{frame}

\subsection{Sensitivity of the results to assumption}
\begin{frame}{\subsecname}
    test \\
    Include Figure 8.
    %\includegraphics[width=0.9\textwidth]{../Figures/Figure_8.png}
\end{frame}

\subsection{Scope and limitations of the data and foootprinting approaches}
\begin{frame}{\subsecname}
    \begin{itemize}
        \item Limitations linked to data Sources
        \item Carbon footprints and individual responsibility
    \end{itemize}
\end{frame}

\subsection{How our estimates compare to earlier work}
\begin{frame}{\subsecname}
\end{frame}

\subsection{Stylized facts on inequality and emissions}
\begin{frame}{\subsecname}
    Stylized facts.
\end{frame}

\subsection{Distributional properties and revenue estimates for a carbon wealh tax}
\begin{frame}{\subsecname}
\end{frame}

\section{Conclusion}
\begin{frame}
    \tableofcontents[currentsection, hideothersubsections, sections=\value{section}]
\end{frame}


\end{document}