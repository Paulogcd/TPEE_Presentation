\documentclass[10pt]{beamer}

\usetheme[progressbar=frametitle]{metropolis}
\usepackage{appendixnumberbeamer}

\usepackage{booktabs}
\usepackage[scale=2]{ccicons}

\usepackage{pgfplots}
\usepgfplotslibrary{dateplot}

\usepackage{xspace}
\newcommand{\themename}{\textbf{\textsc{metropolis}}\xspace}


\title{Carbon Footprint of Capital}
\subtitle{Notes Part 1-2-6}
% \date{\today}
\date{}
\institute{Topics - Fall 2024}
% \titlegraphic{\hfill\includegraphics[height=1.5cm]{logo.pdf}}

\begin{document}

\maketitle

%%%%%
% INTRODUCTION
%%%%%

\begin{frame}{1) Introduction}

    \begin{columns}[t]
        \column{0.4\textwidth}
        \textbf{insert picture}
        
        \column{0.6\textwidth}
        \textbf{Carbon Footprint:}
        
        \textit{The measure of the exclusive total amount of emissions of carbon dioxide that is directly and indirectly caused by an activity or is accumulated over the life-cycle stages of a product.}
        
        \vspace{10pt}
        \textbf{Individual} Carbon Footprint:

        \textit{The carbon footprint associated with an individual’s activities, lifestyle or choices.}

        \vspace{10pt}
        \textbf{Challenge:} \texit{What to include in the carbon footprint?}

        \vspace{10pt}
        \textit{The Consumption-based Approach}
        
    \end{columns}

\end{frame}

\begin{frame}{1) Introduction}



\begin{center}
\textbf{Chancel and Rehm (2024):}

"The Carbon Footprint of Capital: 
\textit{Evidence from France, Germany and the US based on Distributional Environmental Accounts}"
\end{center}

\textbf{Motivations:}

\textit{Individuals are not only responsible for their consumption, but also for the assets they own.}

    \begin{enumerate}
        \item Linking carbon emissions to asset ownership to construct a new framework for individual carbon footprint.
        \item Applying it to France, Germany and the US.
        \item Deriving new stylized facts about emissions inequality in the context of environmental and tax policy.
    \end{enumerate}
    
 \item \textbf{3 Approaches:} Consumption, Ownership and Mixed.

\end{frame}



\begin{frame}{1) Introduction}

 \small \textbf{Key findings:}
  {\footnotesize
    \begin{enumerate}
        \item Carbon inequalities are notable in every approach.
        \item In the ownership approach, the majority of emissions for the wealthiest 10 \percent originates from the assets they own.
        \item Emissions from capital ownership appear to be even more concentrated than capital itself.
    \end{enumerate}
  }
  
  \begin{figure}[h]
    \centering
    \includegraphics[width=1\linewidth]{chancel_fig1.png}
    \label{fig:schnabel}
   \end{figure}

\end{frame}



\begin{frame}{1) Introduction}

\textbf{Policy recommendation:} Tax of 150 euros/dollars per t levied on the carbon content of assets.

 \textbf{Contributions}
    \begin{enumerate}
        \item Proposing a \textbf{new framework} for measuring individual carbon footprint
        \item Studying wealth and income inequality through \textbf{distributional environmental accounts}
        \item Departing from the literature on carbon inequality which focused on consumption-related emissions
        \item Prodiving original data on the carbon content of asset classes
    \end{enumerate}
    
\end{frame}


%%%%
% Literature
%%%%

\begin{frame}{2) Literature Review}

\textit{What makes a good Carbon Footprint estimate?}

{\small
The 2 fundamentals or carbon accounting:
    \begin{enumerate}
        \item \textbf{Comprehensiveness:} measuring both direct and indirect emissions associated with the economic activity
        \item \textnbf{Exclusivity:} no double-counting

    \end{enumerate}
}

So far, the two common ways to measure the carbon footprint have been to focus on \textbf{Iinstitutions} (countries, firms) or \textbf{individuals} (as final consumers).

\end{frame}


\begin{frame}{2) Literature Review}

\textbf{Consumption-based Approaches:} Individuals' consumption guide the resource allocation in the economy
    \begin{itemize}
        \item Underlying assumption: \textit{"Individuals express their preferences through consumption, which sens a signal to producers about what to manufacture and in what quantity."}
        \item The \textbf{"consumer-pays"} principle
    \end{itemize}

    {\small
    \textbf{Advantage:} particularly relevant at the country level (accounts for outsourced emissions)

    \textbf{Drawback:} puts the entire responsibility for all emissions on final consumers (despite market failures: lack of information, agency or alternatives)
    }
    
\end{frame}



\begin{frame}{2) Literature Review}
\textit{Contrasting consumption footprints with the production footprints of firms.}
\vspace{7pt}

\textbf{Production-centered approaches}: Institutional approaches
    \begin{itemize}
        \item Extraction-based approach
        \item Control-based carbon accounting
    \end{itemize}
{\small Critique: \textit{firms operate through human intervention and individuals are behind their behaviors}
} \rightarrow \textbf{Ownership-based approach}
\vspace{10pt}

\textbf{Methods of shared attribution}: split emissions between consumers and firm owners 
{\small Critique: Hard to implement at the individual level} \rightarrow \textbf{Mixed-based approach}
\vspace{10pt}

\textbf{Income-based carbon accounting}

\end{frame}



\begin{frame}

\textit{There already were some attempts at measuring the carbon emissions of individual portfolios. But there exists no consensus regarding these methods and their estimates were not always consistent with aggregate estimates.}
\vspace{10pt}

\textbf{The Distributional National and Environmental Accounts}

\begin{itemize}
    \item Advances in Distributional Accounting
    \item The 2008 revision of the System of National Accounts and the integration of emission data
    \item The DINA Framework: \textit{Reconciling macroeconomic studies (e.g., production, income, wealth) with microeconomic distributional analysis by integrating the study of inequality into the system of national accounts}
\end{itemize}

\end{frame}


%%%%
% DISCUSSION
%%%%

\begin{frame}{6) Discussion}
    \textbf{Successful robustness checks}

    \textbf
    
\end{frame}


\end{document}